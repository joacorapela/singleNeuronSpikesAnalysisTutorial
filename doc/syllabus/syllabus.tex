
\documentclass[10pt]{article}

\title{Syllabus for tutorial on statistical analysis of single-neuron spiking activity}
\author{}

\begin{document}

\section{Content}

The goal of this tutorial is to provide participants with computationaal
experience (e.g., statistics, programming, plotting) to better understand
single-neuron spiking activity.

We will use spikes from one neuron recorded by Cristina Masuzki from the amygdala of a
female mouse while she was interating, in different sessions, with two other
female mice (i.e., female1 and female2). One motivation for this analysis is to
try find features of the recordings from which a downstream neuron could tell
the identity of the interacting mice (female1 or female2) by only looking at
the spiking activity of this neuron.

In the first part of this tutorial we will inspect the recorded data with
different statistical measures (e.g., , inter-spike intervals,
autocorrelations; Section~\ref{sec:descriptive_statistics}), and in the second
part we will attempt to infer properties of the the spiking activity of the
analyzed neuron using models (Section~\ref{sec:inferential_statistics}).

\subsection{Descriptive statistics}

We will begin by plotting 

\subsection{Inferencial statistics}

\end{document}
